\section{The Radical}
The \textbf{(Jacobson) radical} of a ring $R$ is defined to be the left ideal $\J(R)$ given by the intersection of all maximal left ideals of $R$.

\begin{lem} If $E$ is a simple $R$-module, then $\J(R)E=0$. 
\begin{proof}
Since $E$ is simple $E \simeq R/I$ for some maximal left ideal $I$ of $R$. 
Note that $\J(R) \subseteq I$. If $a \in \J(R)$ and $x \in R$ then $a(x+I)=ax + I \in I$.
\end{proof}
\end{lem} 

\begin{thm}[Nakayama's Lemma] Let $E$ be a finitely generated $R$-module. 
If $\J(R)E =E$, $E =0$. 
\begin{proof} 
By induction on the number of generators of $E$. 
Let $x_1, \ldots, x_s$ be such generators.
By assumption there are $a_1, \ldots, a_s \in \J(R)$ such that 
$$x_s=a_1 x_1 + \cdots + a_s x_s \text{.}$$
Therefore there is $a \in \J(R)$ such that $(1+a)x_s$ lies in the module generated by the first $s-1$ generators. 
Moreover $1+a$ is a unit in $R$, otherwise $1+a$ is contained in some maximal ideal (every non-unit is contained in a maximal ideal) but then $1$ belongs to this maximal ideal, since $a \in \J(R)$.
Absurd. 
Hence $x_s$ lies in the module generated by $x_1, \ldots, x_{s-1}$ and by induction $E=0$. 
\end{proof}
\end{thm}

\begin{cor}
\label{naka_cor}
Let $E$ be an $R$-module and $F$ a submodule. If $\J(R)E+F=E$, then $F=E$. 
\begin{proof}
Apply Nakayama's Lemma to $E/F$\footnote{It's possible to prove this result directly applying Zorn's Lemma. Of course Zorn's Lemma is needed in the proof of theorem above too.}.
\end{proof}
\end{cor}

The radical has two important characterizations. 

\begin{thm}
$\J(R)$ is equal to the following: 
\begin{enumerate}
    \item the intersection of all annihilators of simple $R$-modules; 
    \item  the set of elements $x \in R$ such that every element of the form $1 + Rx$ is a unit.
\end{enumerate}
\begin{proof}
Recall that an $R$-module $E$ is simple iff it's isomorphic to $R/I$ for some maximal left ideal $I$ of $R$. 
Then $\Ann_R E \simeq \Ann_R R/I$.  
But $\Ann_R R/I \simeq I$. 
This proves the first statement. 

Let $x \in \J(R)$ and $a \in R$. 
Define $y:=1+ax \in R$. 
Then $R=\J(R)R+Ry$ and, by Corollary \ref{naka_cor}, $R=Ry$. Now, use the analogue for right $R$-modules of Corollary \ref{naka_cor} to get $R=yR$. 
This shows that $1+ax$ is invertible. 
Conversely, assume $1+ax$ is invertible for all $a \in R$ but $x \notin J$ for some maximal left ideal $J$. 
$x+J$ is invertible in $R/J$, so that $1+ax \in J$ for some $a$. 
But then $1 \in J$, absurd. 
\end{proof}
\end{thm}

A ring $R$ is called \textbf{semiprimitive} if $R$ has a faithful semisimple left module. 

\begin{thm} A ring $R$ is semiprimitive iff $\J(R)=0$.
\begin{proof}
Let $E$ be a faithful semisimple $R$-module. $E=\bigoplus E_i$ where each $E_i$ is simple. 
By the first above characterization of $\J(R)$, $\J(R)E=0$. 
Therefore $\J(R) \subseteq \Ann_R E =0$ and $\J(R)=0$. 

Conversely, assume that $\J(R)=0$. Let $\{E_i\}$ be a family of pairwise nonisomorphic $R$-modules such that each isomorphism class of simple $R$-modules is represented. 
Let $E:=\bigoplus E_i$. 
$E$ is semisimple and 
$$\Ann_R E = \bigcap \Ann_R E_i =\J(R)=0\text{.}$$ 
\end{proof}
\end{thm}

In particular, since $\J(R/\J(R))=0$, $R/\J(R)$ is always semiprimitve. 

\subsection*{Exercises} 
\begin{ex} \label{radicalzero} Let $R$ be artinian. Show that its radical is $0$ iff $R$ is semisimple.
\end{ex}

\begin{ex} \label{nilpotent} Prove that if $R$ is artinian, then its radical is nilpotent. (Hint: Nakayama's Lemma). 
\end{ex} 
