
\section{Modules over a Principal Ring} By an \textbf{entire} ring we mean a commutative ring in which $1\neq 0$ and without zero-divisors. 
A \textbf{principal ring} is an entire ring whose ideals are principal.
Let $R$ be a principal rings. Modules are left $R$-modules and linear maps are $R$-linear.

If $F$ is a free module with basis $\{x_i\}_{i \in I}$, the cardinality of $I$ is uniquely determined,  and is called the \textbf{dimension} of $F$. 
Assuming that the dimension of a vector space is a well-defined cardinal number, this is proved by taking a prime element $p$ in $R$: $F/pF$ is a vector space over the field $R/pR$ and the dimension is the same. 

\begin{thm} 
Let $F$ be a free module and $M$ a submodule. 
$M$ is free and $\dim M \le \dim F$. 
\begin{proof} 
We prove the statement by induction on $n=\dim F$\footnote{If the dimension of $F$ is not finite then one enumerates the basis and employs transfinite induction.}. 
If $n = 1$, then $F \simeq R$. 
Thus, $M$ is isomorphic to $0$ or $R$. 

Let's prove the inductive step. 
If $\{x_1, \ldots, x_{n+1}\}$ is a basis of $F$, define $F_n := \<x_1,...,x_n\>$, and let $M_n: = M \cap F_n$. 
$M_n$ is a free module of dimension $ \le n$. 
Now
$$\frac{M}{M_n} = \frac{M}{M \cap F_n} \simeq \frac{M + F_n }{ F_n } \le \frac{F}{ F_n } \simeq R\text{.} $$
By the base step, either $M/M_n$ is isomorphic to $ 0$ or to $R$. 
In the first case, $M = M_n$, and we're done. 
In the second case, by Corollary \ref{corollary} $M = M_n \oplus \<m\>$ for some $m \in M$. 
Thus $M$ is free of dimension $\le n+1$.
\end{proof}
\end{thm}

\begin{cor} Let $E$ be a finitely generated module and $M$ a submodule. 
Then $M$ is finitely generated. 
\begin{proof} If $E=\<v_1, \ldots, v_n\>$ we consider a free module $F$ with basis $\{x_1, \ldots, x_n\}$. 
The map $x_i \mapsto v_i$ is linear. 
The preimage of $M$ in $F$ is free and hence finitely generated so that $M$ is finitely generated too.  
\end{proof}
\end{cor}

A free one-dimensional module is thus isomorphic to $R$ and called \textbf{cyclic}. 

$x$ in a module $M$ is said to be a \textbf{torsion} element if there a nonnull coefficient in $R$ such that $ax=0$. 
Let $M_{\tor}$ be the submodule of torsion elements. 
We say that $M$ is a \textbf{torsion module} if $M=M_{\tor}$ and that $M$ is \textbf{torsion free} if $M_{\tor}=0$. 

\begin{lem} 
If $M$ is a finitely generated torsion free module, then $M$ is free. 
\begin{proof} 
Let $M=\<y_1, \ldots, y_m\>$ and let $\{v_1, \ldots, v_n\}$ be a maximal linearly independent subset of these generators. 
If $y$ is one of these generator there are $a , b_i \in R$ (not all zero) such that 
$$ay + b_1 v_1 + \ldots + b_n v_n =0 \text{.}$$ 
But then $a \neq 0$ and thus $ay \in \<v_1, \ldots, v_n\>$. 
Let $a_j \neq 0$ the correspondent $a$ of each $y_j$, $j=1, \ldots, m$, such that $a_j y_j \in \<v_1, \ldots, v_n\>$. 
Let $a:=a_1 \cdots a_m$ the product of these coefficients. 
$aM$ is contained in $\<v_1, \ldots, v_n\>$ and thus is free. 
But $M$ is torsion free: the map $x \mapsto ax$ is injective so that $M \simeq aM$.
\end{proof}
\end{lem}

\begin{thm} 
Let $E$ be a finitely generated module. 
Then $E/E_{\tor}$ is free. 
Moreover, there is a free submodule $M\le E$ such that $E = E_{\tor} \oplus M$. 
\begin{proof} We prove that $E/E_{\tor}$ is torsion free and we apply the lemma (clearly $E/E_{\tor}$ is finitely generated). 
Let $X:=x+ E_{\tor}$ for some $x \in E$. 
Let $b$ a nonnull coefficient such that $bX=0$. 
Then $bx \in E_{\tor}$ and there is a nonnull $c \in R$ such that $cbx=0$. 
Thus $x \in E_{\tor}$ and $X=0$.  

The map $E \to E/E_{\tor}$ is surjective. 
We can apply Corollary \ref{corollary} to obtain the second part of the statement. 
\end{proof}
\end{thm}

As a corollary we get that the dimension of such submodule $M$ is uniquely determined ($M \simeq E/E_{\tor}$).
This number is called the \textbf{rank} of $E$.

Recall that the \textbf{annihilator} of a subset $S\subseteq E$ is the ideal $\Ann_R S$ of $R$ given by those coefficients that zero all the elements of $S$. 
In particular we have: $\Ann_R x= \<a\>$ and $\Ann_R E= \<b\>$. 
We say that $a$ is a \textbf{period} of $x$ (and $b$ of $E$). 
A period is determined up to multiplication by a unit (whence we shall mostly say \textit{the} period). 
The period of a torsion module $E$ is positive, and it's not $1$ if $E \neq 0$. 

$c\in R$, $c \neq 0$, is called \textbf{exponent} for $x$ (for $E$) if $cx=0$ ($cE=0$). 
Let $E_c$ be the submodule of elements with exponent $c$. 
Let $p$ being prime we denote by $E(p)$ the submodule of elements with exponent a power of $p$. 

\begin{thm}
Let $E$ be a finitely generated nonzero torsion module with period $a=p_1 ^{r_1} \cdots p_n ^{r_n}$. 
Then 
$$E=E(p_1) \oplus \cdots \oplus E(p_n)\text{.}$$
\begin{proof}
We argue inductivey: let $a=bc$ with $b,c$ coprime. 
Let $x , y \in R$ such that $1=xb+yc$. 
We show that $E=E_b \oplus E_c$. 
Let $z \in E_b \cap E_c$. 
Then $z=z(xb+yc)=0$. 
Moreover, let $z =xbz + ycz$. 
$xbz \in E_c$ because $cxbz=axz=0$; $ycz \in E_b$. 
\end{proof}
\end{thm}

$y_1, \ldots, y_m \in E$ are \textbf{independent} if whenever the expression $a_1 y_1 + \cdots + a_m y_m$ with coefficient in $R$ is zero then $a_i y_i =0$ for all $i$. 
This is equivalent to require that the module $\< y_1 , \ldots , y_m\>$ has decomposition 
$$\< y_1 , \ldots , y_m\> = \< y_1\> \oplus \cdots \oplus \< y_m\>  \text{.}$$ 

\begin{lem} Let $E$ be a torsion module with exponent $p^r$ and $ x \in E$ an element with period $p^r$. 
Let $Y_1, \ldots, Y_m$ independent elements of $E/\<x\>$. 
There is a representative $y_i$ of $Y_i$ with same period such that $x, y_1, \ldots, y_m$ are independent.

\begin{proof}
Let $Y$ have period $p^n$ and $y$ be a representative. 
Since $p^n Y=0$, then $p^n y \in \<x\>$: $p^n y =p^s c x$ with $p$ and $c$ coprime and $s \le r$. 
If $s=r$ we are done. 
If $s <r$, then $p^s cx$ has period $p^{r-s}$ and $y$ has period $p^{n+r-s}$. 
Since $p^r$ is an exponent of $E$, $n+r-s \le r$. 
Thus $n \le s$; $y-p^{s-n} c x$ is a representative of $Y$ with same period. 
As to the last statement, suppose that 
$$ax + a_1 y_1 + \cdots + a_m y_m=0\text{.}$$ 
Then $a_1 Y_1 + \cdots a_m Y_m=0$. 
By hypothesis $a_i Y_i=0$. 
If $p^{r_i}$ is the period of $Y_i$, then $p^{r_i}$ divides $a_i$. 
Then $a_i y_i =0$ and finally $ax=0$. 
\end{proof}
\end{lem}

Each $E(p)$ can be written as a direct sum $$E(p) = \frac{R}{\< p^{r_1}\>} \oplus \cdots \oplus \frac{R}{\< p^{r_s}\>} $$
with $1 \ge r_1 \ge \cdots \ge r_s$. 
Observe that $E(p)=:E$ is finitely generated. 
Let $x_1 \in E$ with period $p^{r_1}$ for some prime $p$ and maximal $r_1$. 
Consider $E/\<x_1\>$. 
$(E/\<x_1\>)_p$ is a vector space over $R/pR$, whose dimension is strictly less than the dimension of $E_p$. 
If $Y_1, \ldots, Y_m$ are linearly independent elements of $(E/\<x_1\>)_p$ then lemma above implies that $\dim E_p \ge m+1$: we can always find an element of $\<x_1\>$ with period $p$ independent of $y_1, \ldots, y_m$. 
Then we go on by induction: there are $X_2, \ldots, X_s$ with periods $p^{r_2}, \ldots, p^{r_s}$ such that $r_2 \ge \ldots \ge r_s$. 
By lemma above there are representatives $x_2, \ldots, x_s$ with period $p^{r_i}$ and such that $x_1, \ldots, x_s$ are independent. 
Since $r_1$ is maximal, $r_1 \ge r_2$. 


The decomposition is essentially unique, as the following result (which we don't prove) illustrates. 

\begin{thm} Let $E$ be a finitely generated nonzero torsion module. 
Then 
$$E\simeq \frac{R}{\<q_1\>}\oplus \cdots \oplus \frac{R}{\<q_r\>}$$
where each $q_i$ is a nonzero element of the ring. 
Moreover $q_1 \mid q_2 \mid \cdots \mid q_r$ and the sequence of ideals $\<q_1\>, \ldots, \<q_r\>$ is uniquely determined. 
\end{thm}

The ideals $\<q_1\>, \ldots, \<q_r\>$ are called \textbf{invariants} of $E$. 

\subsection*{Exercises}
\begin{ex}
Deduce the structure theorem for finite abelian groups:  every finite abelian group can be expressed as the direct sum of cyclic subgroups of prime-power order. 
(Hint: a group is a $\ZZ$-module).
\end{ex}
