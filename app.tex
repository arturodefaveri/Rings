\appendix 
\section{Jordan Normal Form}
Module theory provides a neat way to prove that every square matrix with coefficients in an algebraically closed field is similar to a Jordan matrix.  

\begin{thm}
Let $k$ be an algebraically closed field, and let $E$ be a $k$-vector space of dimension $n>0$. 
Let $A \in \End_{k}(E)$. 
There is a basis of $E$ over $k$ such that with respect to this basis $A$ is in normal form. 

\begin{proof}
$E$ is a $k[A]$-module, and, thanks to the map $k[x] \to k[A]$ that sends $x$ into $A$, a $k[x]$-module as well. 
Note that $k[x]$ is a principal ring. The above map is surjective and has nontrivial kernel generated by a polynomial $q(x)$ (called \textbf{minimal polynomial} of $A$). 
Since $0=q(A)E=q(x)E$, $q$ is the period of $E$. 
$q$ has a factorization $p_1 ^{r_1} \cdots p_s ^{r_s}$ into distinct prime powers. 
Hence $E = E(p_1) \oplus \cdots \oplus E(p_s)$. 

Each $E(p)$ can be written as a direct sum of submodules isomorphic to $k[x]/\<p^r\>$ for some irreducible polinomial $p(x)=(x-\alpha)$ and some $r \ge 1$. 
We investigate the structure of the submodule $k[x]/\<p^r\>$. Let $v \in E$ with period $p^r$. 
We show that the elements 
$$\{v, (x-\alpha)v, \ldots, (x-\alpha)^{r-1}v\}$$ 
(or better
$$\{v, (A-\alpha)v, \ldots, (A-\alpha)^{r-1}v\}\text{)}$$
are a basis. 
They are linearly independent over $k$, because the dependence between 
$$v, (A-\alpha)v, \ldots, (A-\alpha)^{r-1}v$$ 
implies that of $v, Av, \ldots, A^{r-1}v$. 
But then there is a polynomial $g(x)$ with degree less than $r$ such that $g(A)=0$. 
Since $\dim_k(k[x]/\<p^r\>)=r $ the set is a basis (and is the basis by which $A$ is in the required form).  
\end{proof}
\end{thm}

\section{The Tensor Product} 
Fix a pair of $R$-modules $E_1,E_2$. 
If $F$ is an $R$-module, the set $\Bil(E_1,E_2;F)$ of bilinear maps from $E_1 \times E_2$ to $F$ forms an $R$-module. 

Now, let $\textsf{C}$ be the category whose objects are bilinear maps $f: E_1 \times E_2 \to F$. We define an arrow from $f: E_1 \times E_2 \to F$ to $g: E_1 \times E_2 \to G$ to be a $R$-linear map $h: F \to G$ such that the following diagram commute: 

\begin{center}
\begin{tikzcd}[row sep=tiny]
                        & F \arrow[dd, "h"] \\
E_1 \times E_2 \arrow[ur, "f"] \arrow[dr, "g"] & \\
                        & G 
\end{tikzcd}
\end{center}

The \textbf{tensor product} of $E_1$ and $E_2$ is a universal object in the category $\textsf{C}$, i.e. it comprises an $R$-module $E_1 \otimes E_2$ and a bilinear map $\phi : E_1 \times E_2 \to E_1 \otimes E_2$ such that for every object of $\sf{C}$ $g:E_1 \times E_2 \to G$ there is a unique $g_{\star}: E_1 \otimes E_1 \to G$ making the following diagram commute 

\begin{center}
\begin{tikzcd}[row sep=tiny]
                        & E_1 \otimes E_2 \arrow[dd, "g_{\star}"] \\
E_1 \times E_2 \arrow[ur, "\phi"] \arrow[dr, "g"] & \\
                        & G 
\end{tikzcd}
\end{center}

We now show that such an object exists. 
Uniqueness is clear by universality. 

\begin{thm}
$\sf{C}$ admits a universal object. 
\begin{proof}
Let $M$ be the free $R$-module generated by the pairs $(e_1, e_2) \in E_1 \times E_2$ and let $M/\sim$ be the quotient of $M$ modulo the following relations: 
$$(e_1 + e'_1, e_2) \sim (e_1, e_2) + (e'_1, e_2) \quad (e_1, e_2+ e'_2) \sim (e_1, e_2) + (e_1, e'_2) \quad (ae_1, e_2) \sim (e_1, ae_2) \sim a(e_1, e_2)$$ 
for every $a \in R$. 
Define $E_1 \otimes E_2 :=M/\sim$ and $\phi$ to be the composition of the inclusion $E_1 \times E_2 \hookrightarrow M$ with the canonical map $M \to M/\sim$. 
Now, consider the diagram (where $\bar{g}$ is induced by $g$ by linearity):
\begin{center}
\begin{tikzcd}
E_1 \times E_2 \arrow[r, hook] \arrow[dr, "g"]
& M \arrow[r] \arrow[d, "\bar{g}"] 
& E_1 \otimes E_2 \arrow[dl, dashrightarrow, "g_{\star}"]\\
& G 
\end{tikzcd}
\end{center}
Since $\bar{g}$ takes the value $0$ on the submodule generated by $\sim$, the universal property of factor modules assures that there is a unique $g_{\star}$ that gets the job done. 
\end{proof}
\end{thm} 

\section{Solutions to Selected Exercises} 
\ref{free1}. Taken $n \in N$, by surjectivity there are $k \in K$, $l \in L$ such that $n = f(k+l)=f(k)+f(l)$. Hence $N=f(K) + f(L)$. We show that $f(K) \cap f(L)= 0$. Let $x \in f(K) \cap f(L)$; in particular $x=f(k)=f(l)$ for some $k \in K$, $l \in L$. 
$$0=x-x=f(l)-f(k)=f(l-k)$$ 
implies that $l-k \in \ker f=K \cap L$. 
$l \in L$ and $(l-k)+k \in K$ hence $f(l)=0$ and $x=0$.

\ref{boolean} Since $e_1, \ldots, e_n$ are idempotent, orthogonal $R \simeq Re_1 \oplus \cdots \oplus Re_n$. 
But since $\{e_1, \ldots, e_n\}$ are primitive then each $Re_i$ is isomorphic to $\ZZ_2$. 
Suppose not, then there is $x \in Re_i$ such that $x \neq 0,1$. 
But a subring of a boolean ring is a boolean ring, hence $x^2 =x$. 
This implies that $Re_i \simeq Rx \oplus R(1-x)$ and $e_i$ is not primitive. 
Therefore $R \simeq \ZZ_2 \oplus \cdots \oplus \ZZ_2$. 
Now, let $a \in R$ as above. 
$a=a_1 e_1 + \cdots a_n e_n$ with each $a_i$ equal to $0$ or to $1$. 
This proves our statement.

\ref{artinian} The family of simple $R$-modules is nonempty ($R$ itself belongs to it).
Therefore there is a minimal simple $R$-module $E$. 
For every $a \in R$,  $aE$ is still a $R$-module. 
By simplicity either $aE \simeq E$ or $aE=0$. 
In any case the sum $I:=\sum_{a \in R} aE$ is a sum of simple modules. 
$I$ is a two-sided ideal of $R$.
Since $R$ is simple, $R\simeq I$. 

\ref{wedderburn} Let $\{x_1, \ldots, x_n\}$ be a basis of $E$ as a $D$-space. Let $f \in \End_D(E)$. 
Jacobson's Density Theorem there is $a \in R$ such that $ax_i=f(x_i)$. 
Then the assignment $R \to \End_D(E)$ is surjcetive. 
By faithfulness is injective. 

\ref{radicalzero} By definition the radical of $R$ is the intersection of all maximal left ideals. 
Since $R$ is artinian this intersection is finite: $\J(R):= I_1 \cap \cdots \cap I_k$ (otherwise there'd be an infinite descending sequence of ideals). 
If $\J(R)=0$, the map 
$$R \to \bigoplus_i R/I_i$$ 
is injective. 
Since $R/I_i$ is simple $R$ is a submodule of a semisimple module, whence semisimple. 
Conversely, if $R$ is semisimple, then $R\simeq \bigoplus_i R/I_i$ where each $I_i$ is a maximal left ideal. 
Since this map is injective its kernel $\bigcap_i I_i$ is zero. 
This implies that $\J(R)=0$.

\ref{nilpotent} Let $N:=\J(R)$. 
$N \supset N^2 \supset \cdots$ is a descending sequence of left ideals. 
Let $N^{\infty}$ be its limit. 
We prove that  $N^{\infty}= 0$.  
There is $r$ such that $N^{\infty}=N^r$. 
Suppose that $N^r \neq 0$. 
The set $S$ of ideals $L \subset N^{\infty}$ such that $N^{\infty}L \neq 0$ is nonempty: $N^r N = N^{r+1} \neq 0$. 
Let $L$ be a minimal such ideal. 
By minimality $L$ is finitely generated. 
By Nakayama's Lemma $N(N^rL)=N^rL=0$, contradiction. 
