\section{Caveats} 
We assume that the reader possesses elementary notions about rings and modules. 

These are notes about r\textit{i}ngs, not about rngs: we tacitly assume that every ring has a unit. 
Moreover the product $\cdot$ is associative by hypothesis. 
Normally we don't require that $1\neq0$. 
This trick allows us to treat $0$ as a r\textit{i}ng (and a morphism) without being inconsistent. 

We won't bother with avoiding sloppy notation: $1$ is the unit of every ring as well as the identity function between any two objects; $0$ is the zero of every etc.

Throughout our discussion we shall call a left module over a ring $R$, simply a $R$-module. The reader may generalize our treatment to the case of right modules. 

Alas, lots of important rings, such as local or artinian rings, are relegated to exercises. 
Our apologies.

Last but not least, these notes are far from being original. 
They just fill in the details of lectures given by Prof Yu Chen at Torino. 
For the most it's just been copying and pasting from classic books such as\footnote{However, if you find a typo or a mathematical error -- it's more than possible! -- you can email me at \verb|arturo dot defaveri at edu dot unito dot it}: 
\begin{itemize}
    \item Serge Lang, \emph{Algebra}
    \item Joseph J. Rotman, \emph{Advanced Modern Algebra} 
    \item Frank W. Anderson \& Kent R. Fuller, \emph{Rings and Categories of Modules}
\end{itemize}

Here we list a couple of important things that don't fit naturally the subsequent topics. 

\subsection*{Exercises}

\begin{ex} If $E$ is an abelian group, the set of endomorphisms $\End(E)$ is a ring with pointwise addition and composition. 
Let $E$ be an abelian group. 
Prove that $E$ is a $R$-module iff there is a ring morphism 
$$f: R \to \End(E)\text{.}$$
Deduce that a $R$-module is also an $S$-module whenever there is a ring morphism $g: S \to R$. 
\end{ex}

\begin{ex}
Let $R$ be a ring. 
The ring $R^{\op}$ has same carrier and sum as $R$ but the products are reversed. 
Prove that $R^{\op} \simeq \End_R(R)$. 
(If $E$ is an $R$-module -- and $R$ is -- $\End_R(E)$ is still a ring as above). 
\end{ex}
